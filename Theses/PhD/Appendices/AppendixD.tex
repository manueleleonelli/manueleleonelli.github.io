% Appendix Template

\chapter{Computer Code} % Main appendix title

\label{appendixD} % Change X to a consecutive letter; for referencing this appendix elsewhere, use \ref{AppendixX}

\lhead{Appendix D. \emph{Computer Code }} % Change X to a consecutive letter; this is for the header on each page - perhaps a shortened title


\noindent Computation of the highest index in each parent set of a utility node.\\
\textbf{Inputs}: \verb|PiU::table|, parent sets of utility nodes;
\verb|m::integer|, number of utility nodes.\\
\textbf{Output}: \verb|J::list|.

\vspace{-0.4cm}
\begin{alltt}
CompJ := \textbf{proc}(PiU,m) \textbf{local} i,j:
\textbf{for} j  \textbf{to} m \textbf{do} J[j] := max(PiU[J]) \textbf{end do}:
\textbf{return} convert(J,list):
\textbf{end proc}:
\end{alltt}


\noindent Computation of the indices of the argument of the expected utility at step $i$. \\
 \textbf{Inputs}: \verb|PiU::table|; \verb|PiV::table|, parent sets of random nodes; \verb|i::integer|;\\
\verb|n::integer|, number of random nodes; \verb|J::list|.\\
\textbf{Output}: \verb|Bi[i]::set|.

\vspace{-0.4cm}
\begin{alltt}
CompBi := \textbf{proc}(PiU,PiV,i,n,J) \textbf{local} Bi, part, j:
Bi[i], part := \{\},\{\}:
\textbf{for} j \textbf{from} i \textbf{to} n \textbf{do}
  part := part \textbf{union} \{j\}:
  \textbf{if} member(j,V) \textbf{then} Bi[i] := Bi[i] \textbf{union} PiV[j] \textbf{end if}:
  \textbf{if} member(j,J,'l') \textbf{then} Bi[i] := Bi[i] \textbf{union} PiU[l] \textbf{end if}:
\textbf{end do}:
Bi[i] := Bi[i] \textbf{minus} part:
\textbf{return} Bi[i]:
\textbf{end proc}:
\end{alltt}


Initialisation of an MID.\\
\textbf{Inputs}: \verb|p::table|, probability vectors; \verb|psi::table|, utility vectors; \\
\verb|PiV::table|; \verb|PiU::table|; \verb|n::integer|; \verb|m::integer|.\\
\textbf{Outputs}: \verb|J::list|; \verb|Bi::table|; \verb|u::table|, expected utility vectors.

\vspace{-0.4cm}
\begin{alltt}
Initialise := \textbf{proc}(p, psi, PiV, PiU, n, m) \textbf{local} J, i, Bi, u:
J := CompJ(PiU, m):
\textbf{for} i \textbf{to} n \textbf{do}  Bi[i] := CompBi(PiU, PiV, i, n, J) \textbf{end do}:
Bi[n+1], u[n+1] := \{\}, []:
\textbf{return} J, Bi, u:
\textbf{end proc}:
\end{alltt}


\noindent EUDuplication of a utility vector and an expected utility vector.\\
\textbf{Inputs}: \verb|u::table|; \verb|psi::table|; \verb|j::integer|; \verb|PiV::table|; \verb|PiU::table|;
\verb|r::table|, size of the decision and sample spaces; \verb|Bi::table|; \verb|J::list|.\\
\textbf{Outputs}: \verb|utemp::list|, EUDuplicated version of \verb|u|; \verb|psitemp::list|, EUDuplicated version of \verb|psi|.

\vspace{-0.4cm}
\begin{alltt}
EUDuplicationPsi := \textbf{proc}(u, psi, j, PiV, PiU, r, Bi, J)
\textbf{local} i, uprime, psiprime, psitemp, utemp, x, sx, y, h, z:
i, uprime, psiprime:= max(PiU[j]), [], []: 
psitemp, utemp:= psi[j], u[i+1]:
\textbf{for} x \textbf{from} max(Bi[i+1], PiU[j]) \textbf{by} -1 \textbf{to} 1 \textbf{do}
  \textbf{if} member(x, (PiU[j] \textbf{union} Bi[i+1]) \textbf{minus} (PiU[j] \textbf{intersect} Bi[i+1])) 
    \textbf{then} sx := 1:
    \textbf{for} y \textbf{from} x+1 \textbf{to} max(Bi[i+1], PiU[j]) \textbf{do}
      \textbf{if} member(y, Bi[i+1] \textbf{union} PiU[j]) \textbf{then} sx := sx*r[y] \textbf{end if}
    \textbf{end do}:
    \textbf{if} member(x, Bi[i+1]) \textbf{then} \textbf{for} l \textbf{to} Size(psitemp)[2]/sx \textbf{do} 
      \textbf{for} z \textbf{to} r[x] \textbf{do}
      psiprime := [op(psiprime), op(convert(psitemp, list)
      [(l-1)*sx+1 .. l*sx])]
      \textbf{end do} \textbf{end do}:
      psitemp, psiprime := psiprime, []:
    \textbf{elif} member(x, PiU[j]) \textbf{then} \textbf{for} l \textbf{to} Size(utemp)[2]/sx \textbf{do}
     \textbf{for} z \textbf{to} r[x] \textbf{do} 
     uprime := [op(uprime),op(convert(utemp, list)[(l-1)*sx+1..l*sx])]
     \textbf{end do} \textbf{end do}:
     utemp,  uprime := uprime, []:
\textbf{end if} \textbf{end if} \textbf{end do}:
\textbf{return} utemp, psitemp:
\textbf{end proc}:
\end{alltt}


EuMultiSum between an expected utility vector and a utility vector. \\
\textbf{Inputs}: \verb|u::table|; \verb|psi::table|; \verb|j::integer|; \verb|PiV::table|; \verb|PiU::table|;
 \verb|r::table|; \verb|Bi::table|; \verb|J::list|.\\
\textbf{Outputs}: \verb|ut::list|, expected utility vector after an EUMultiSum.  

\vspace{-0.4cm}
\begin{alltt}
EUMultiSum := \textbf{proc}(u, psi, j, PiV, PiU, r, Bi, J) 
\textbf{local} i, uprime, psiprime, ut;
i := max(PiU[j]);
uprime, psiprime:=EUDuplicationPsi(u, psi, j, PiV, PiU, r, Bi, J);
uprime := convert(uprime, list);
\textbf{if} Size(uprime)[1] = 0 \textbf{then}  ut := k[j]*~psiprime 
\textbf{else} ut := h*~k[j]*~psiprime*~uprime+~uprime+~k[j]*~psiprime \textbf{end if};
\textbf{return} ut: 
\textbf{end proc}:
\end{alltt}


\noindent EUDuplication of a probability vector and an expected utility vector. \\
\textbf{Inputs}: \verb|u::table|; \verb|p::table|; \verb|i::integer|; \verb|PiV::table|; \verb|PiU::table|; 
  \verb|r::table|;\\
   \verb|Bi::table|; \verb|J::list|.\\
  \textbf{Outputs}: \verb|utemp::list|, EUDuplicated version of \verb|u|;
\verb|ptemp::list|, EUDuplicated version of \verb|p|.

\vspace{-0.4cm}
\begin{alltt}
EUDuplicationP := \textbf{proc}(u, p, i, PiV, PiU, r, Bi, J)
 \textbf{local} uprime, pprime, ptemp, utemp, x, sx, y, l, z:
uprime, pprime, ptemp, utemp   := [], [],  p[i], u[i+1]:
Uni:= Bi[i+1] \textbf{union} PiV[i] \textbf{union} PiU[j];
\textbf{if} member(i, J) \textbf{then} member(i, J, 'j');
  \textbf{for} x \textbf{from} max(Uni) \textbf{by} -1 \textbf{to} 1 \textbf{do}
    \textbf{if} member(x, Uni \textbf{minus} ((Bi[i+1] \textbf{union} PiU[j]) \textbf{intersect} 
    (PiV[i] \textbf{union} \{i\}))) \textbf{then} 
      sx := 1;
      \textbf{for} y \textbf{from} x+1 \textbf{to} max(Uni) \textbf{do}
        \textbf{if} member(y, Uni) \textbf{then}  sx := sx*r[y] \textbf{end if}
      \textbf{end do}; 
      \textbf{if} member(x, Bi[i+1] \textbf{union} PiU[j]) \textbf{then}
        \textbf{for} l \textbf{to} Size(ptemp)[2]/sx \textbf{do} \textbf{for} z \textbf{to} r[x] \textbf{do} 
            pprime := [op(pprime), op(convert(ptemp, list)
            [(l-1)*sx+1 .. l*sx])]
        \textbf{end do} \textbf{end do};
        ptemp,  pprime := pprime, [];
      \textbf{elif} member(x, PiV[i]) \textbf{then} 
        \textbf{for} l \textbf{to} Size(utemp)[2]/sx \textbf{do} \textbf{for} z \textbf{to} r[x] \textbf{do} 
            uprime := [op(uprime), op(convert(utemp, list)
            [(l-1)*sx+1 .. l*sx])] 
        \textbf{end do} \textbf{end do};
        utemp, uprime := uprime, []:
\textbf{end if} \textbf{end if} \textbf{end do};
\textbf{else} \textbf{for} x \textbf{from} max(Bi[i+1], PiV[i]) \textbf{by} -1 \textbf{to} 1 \textbf{do}
    \textbf{if} member(x, (Bi[i+1] \textbf{union} PiV[i]) \textbf{minus} (Bi[i+1] \textbf{intersect} 
    (PiV[i] \textbf{union} \{i\}))) \textbf{then}
      sx := 1;
      \textbf{for} y \textbf{from} x+1 \textbf{to} max(Bi[i+1], PiV[i]) \textbf{do}
        \textbf{if} member(y, Bi[i+1] \textbf{union} PiV[i]) \textbf{then} sx := sx*r[y] \textbf{end if} 
      \textbf{end do};
      \textbf{if} member(x, Bi[i+1]) \textbf{then} 
        \textbf{for} l \textbf{to} Size(ptemp)[2]/sx \textbf{do} \textbf{for} z \textbf{to} r[x] \textbf{do} 
            pprime := [op(pprime), op(convert(ptemp, list)
            [(l-1)*sx+1 .. l*sx])]
        \textbf{end do} \textbf{end do};
        ptemp, pprime := psiprime, [];
      \textbf{elif} member(x, PiV[i]) \textbf{then} \textbf{for} l \textbf{to} Size(utemp)[2]/sx \textbf{do}
       \textbf{for} z \textbf{to} r[x] \textbf{do}
            uprime := [op(uprime), op(convert(utemp, list)
            [(l-1)*sx+1 .. l*sx])]
        \textbf{end do} \textbf{end do};
        utemp, uprime := uprime, []:
\textbf{end if} \textbf{end if} \textbf{end do} \textbf{end if}:
\textbf{return} utemp, ptemp:
\textbf{end proc}:
\end{alltt}


\noindent EUMarginalisation over a sample space. \\
\textbf{Inputs}: \verb|u::table|; \verb|p::table|; \verb|i::integer|; \verb|PiV::table|; \verb|PiU::table|; 
  \verb|r::table|;\\
   \verb|Bi::table|; \verb|J::list|.\\
\textbf{Outputs}: \verb|ut::list|, expected utility vector after EUMarginalisation.


\vspace{-0.4cm}
\begin{alltt}
EUMarginalisation := \textbf{proc}(u, p, i, PiV, PiU, r, Bi, J)
\textbf{local} uprime, pprime, rows, cols, l, one, zero, row, Iprime, ut;
uprime, pprime := EUDuplicationP(u, p, i, PiV, PiU, r, Bi, J);
pprime, rows, cols := convert(pprime, list), 1,  Size(pprime)[2]; 
\textbf{for} l \textbf{to} i-1 \textbf{do} \textbf{if} member(l, PiV[i]) \textbf{then}
  rows := rows*r[l] \textbf{end if} \textbf{end do};
one := convert(Vector(r[i], 1), list): 
zero := convert(Vector(cols, 0), list):
row, Iprime := [op(zero), op(one)],  [op(one)]:
\textbf{for} l \textbf{to} rows-1 \textbf{do} Iprime := [op(Iprime), op(row)] \textbf{end do};
Iprime := Matrix(rows, cols, [op(Iprime), op(one)]);
ut := convert(Multiply(Iprime, Vector[column](uprime*~pprime)), list);
\textbf{return} ut:
\textbf{end proc}:
\end{alltt}


\noindent Identification of an optimal policy (at random). \\
\textbf{Inputs}: \verb|r::table|; \verb|i::integer|, index of the decision variable,
 \verb|t::integer|, number of random draws. \\
\textbf{Outputs}: \verb|maxi::vector|, optimal decisions.

\vspace{-0.4cm}
\begin{alltt}
Maximise := \textbf{proc}(r, i, t) \textbf{local} maxi, l: 
maxi := Vector(t, 0):
\textbf{for} l \textbf{to} t \textbf{do} 
  maxi[l] := RandomTools[Generate](integer(range = 1 .. r[i])) \textbf{end do}:
\textbf{return} maxi:
\textbf{end proc}:
\end{alltt}

\noindent EUMaximisation over a decision space. \\
\textbf{Inputs}: \verb|u::table|; \verb|i::integer|; \verb|r::table|. \\
\textbf{Outputs}: \verb|u[i]::list|, expected utility vector after EUMaximisation.

\vspace{-0.4cm}
\begin{alltt}
EUMaximisation := \textbf{proc}(u, i, r) \textbf{local} opt, Istar, j, l, zero;
opt := Maximise(r, i, Size(u[i+1])[2]/r[i]);
zero := convert(Vector(Size(u[i+1])[2], 0), list);
Istar := [];
\textbf{for} j \textbf{to} Size(u[i+1])[2]/r[i] \textbf{do} \textbf{for} l \textbf{to} r[i] \textbf{do}
    \textbf{if} opt[j] = l \textbf{then}  Istar := [op(Istar), 1]  
    \textbf{else} Istar := [op(Istar), 0] \textbf{end if} 
 \textbf{end do}; 
  \textbf{if} j < Size(u[i+1])[2]/r[i] \textbf{then}
     Istar := [op(Istar), op(zero)] \textbf{end if} \textbf{end do};
Istar := Matrix(Size(u[i+1])[2]/r[i], Size(u[i+1])[2], Istar);
u[i] := convert(Multiply(Istar, Vector[column](u[i+1])), list); 
\textbf{return} u[i]:
\textbf{end proc}:
\end{alltt}


\noindent Symbolic evaluation algorithm for an MID. \\
\textbf{Inputs}: \verb|p::table|; \verb|psi::table|; \verb|PiV::table|; \verb|PiU::table|; \verb|n::integer|;\\
 \verb|m::integer|; \verb|De::set|, index set of the decision variables; \verb|V::set|, index set of the random variables; \verb|r::table|. \\
\textbf{Output}: \verb|eu::table|, expected utility vectors.

\vspace{-0.4cm}
\begin{alltt}
SymbolicExpectedUtility := \textbf{proc}(p, psi, PiV, PiU, n, m, De, V, r)
\textbf{local} J, Bi, utemp, i, j, ceu,best;
with(LinearAlgebra): with(ArrayTools):
J, Bi, eu := Initialise(p, psi, PiV, PiU, n, m);
j := m;
\textbf{for} i \textbf{from} n \textbf{by} -1 \textbf{to} 1 \textbf{do} \textbf{if} j = 0 \textbf{then} \textbf{if} member(i, De) \textbf{then}
      eu[i] := EUMaximisation(eu, i, r) 
    \textbf{else} eu[i] := EUMarginalisation(eu, p, i, PiV, PiU, r, Bi, J) 
    \textbf{end if}; 
  \textbf{else} \textbf{if} J[j] = i \textbf{then}  \textbf{if} member(i, De) \textbf{then} 
        utemp[i+1] := EUMultiSum(eu, psi, j, PiV, PiU, r, Bi, J); 
        eu[i] := EUMaximisation(utemp, i, r) 
      \textbf{else}
        utemp[i+1] := EUMultiSum(eu, psi, j, PiV, PiU, r, Bi, J); 
        eu[i] := EUMarginalisation(utemp, p, i, PiV, PiU, r, Bi, J)
      \textbf{end if};
      j := j-1
    \textbf{else} \textbf{if} member(i, De) \textbf{then} eu[i] := EUMaximisation(eu, i, r)
      \textbf{else} eu[i] := EUMarginalisation(eu, p, i, PiV, PiU, r, Bi, J)
\textbf{end if} \textbf{end if} \textbf{end if} \textbf{end do};
\textbf{return} eu: 
\textbf{end proc}:
\end{alltt}

\noindent Consider the MID in Figure \ref{fig-ex} with $n=6$ variables (decision or random nodes) and $m=3$ utility nodes.\\
Definition of the MID number of variables and  utility nodes.\\
\verb|n := 6: m := 3:|\\
\verb|V|, contains the indices of random nodes and \verb|De|  those of the decision nodes.\\
\verb|V := {2, 3, 5, 6}: De := {1, 4}:|\\
Conditional probabilities.\\
\verb|p[6] := [p6111, p6011, p6101, p6001, p6110, p6010, p6100, p6000]:|\\
\verb|p[5] := [p5111, p5011, p5101, p5001, p5110, p5010, p5100, p5000]:|\\
\verb|p[3] := [p3111, p3011, p3101, p3001, p3110, p3010, p3100, p3000]:|\\
\verb|p[2] := [p211, p201, p210, p200]:|\\
Utility parameters.\\
\verb|psi[1] := [psi11, psi10]:|\\
\verb|psi[2] := [psi21, psi20]:|\\
\verb|psi[3] := [psi311, psi301, psi310, psi300]:|\\
Parents of random nodes.\\
\verb|PiV[2] := {1}:  PiV[3] := {1, 2}:  PiV[5] := {3, 4}: PiV[6] := {4, 5}:|\\
Parents of utility nodes.\\
\verb|PiU[1] := {3}: PiU[2] := {5}: PiU[3] := {4, 6}:|\\
Number of levels of the variables.\\
\verb|r[1] := 2: r[2] := 2: r[3] := 2: r[4] := 2: r[5] := 2: r[6] := 2:| \\
Computation of the expected utility vectors.\\
\verb|eu := SymbolicExpectedUtility(p, psi, PiV, PiU, n, m, De, V, r):|

\noindent Example of the output of \verb|SymbolicExpectedUtility|, \verb|eu[1]|:
\begin{alltt}
[((k[1]*psi11+h*k[1]*psi11*((k[2]*psi21+h*k[2]*psi21*
(p6010*psi300*k[3]+p6110*psi310*k[3])+k[3]*psi300*p6010
+k[3]*psi310*p6110)*p5101+(k[2]*psi20+h*k[2]*psi20*
(p6000*psi300*k[3]+p6100*psi310*k[3])+k[3]*psi300*p6000
+k[3]*psi310*p6100)*p5001)+(k[2]*psi21+h*k[2]*psi21*
(p6010*psi300*k[3]+p6110*psi310*k[3])+k[3]*psi300*p6010
+k[3]*psi310*p6110)*p5101+(k[2]*psi20+h*k[2]*psi20*
(p6000*psi300*k[3]+p6100*psi310*k[3])+k[3]*psi300*p6000
+k[3]*psi310*p6100)*p5001)*p3110+(k[1]*psi10+h*k[1]*psi10*
((k[2]*psi21+h*k[2]*psi21*(p6011*psi301*k[3]
+p6111*psi311*k[3])+k[3]*psi301*p6011+k[3]*psi311*p6111)*p5110+
(k[2]*psi20+h*k[2]*psi20*(p6001*psi301*k[3]+p6101*psi311*k[3])
+k[3]*psi301*p6001+k[3]*psi311*p6101)*p5010)+
(k[2]*psi21+h*k[2]*psi21*(p6011*psi301*k[3]+p6111*psi311*k[3])
+k[3]*psi301*p6011+k[3]*psi311*p6111)*p5110+
(k[2]*psi20+h*k[2]*psi20*(p6001*psi301*k[3]+p6101*psi311*k[3])
+k[3]*psi301*p6001+k[3]*psi311*p6101)*p5010)*p3010)*p210+((k[1]*
psi11+h*k[1]*psi11*((k[2]*psi21+h*k[2]*psi21*(p6010*psi300*
k[3]+p6110*psi310*k[3]+k[3]*psi300*p6010+k[3]*psi310*p6110)*p5101
+(k[2]*psi20+h*k[2]*psi20*(p6000*psi300*k[3]+p6100*psi310*k[3])
+k[3]*psi300*p6000+k[3]*psi310*p6100)*p5001)+
(k[2]*psi21+h*k[2]*psi21*(p6010*psi300*k[3]+p6110*psi310*k[3])
+k[3]*psi300*p6010+k[3]*psi310*p6110)*p5101+
(k[2]*psi20+h*k[2]*psi20*(p6000*psi300*k[3]+p6100*psi310*k[3])
+k[3]*psi300*p6000+k[3]*psi310*p6100)*p5001)*p3100+(k[1]*psi10
+h*k[1]*psi10*((k[2]*psi21+h*k[2]*psi21*(p6011*psi301*k[3]
+p6111*psi311*k[3])+k[3]*psi301*p6011+k[3]*psi311*p6111)*p5110+
(k[2]*psi20+h*k[2]*psi20*(p6001*psi301*k[3]+p6101*psi311*k[3])
+k[3]*psi301*p6001+k[3]*psi311*p6101)*p5010)+
(k[2]*psi21+h*k[2]*psi21*(p6011*psi301*k[3]+p6111*psi311*k[3])
+k[3]*psi301*p6011+k[3]*psi311*p6111)*p5110+
(k[2]*psi20+h*k[2]*psi20*(p6001*psi301*k[3]+p6101*psi311*k[3])
+k[3]*psi301*p6001+k[3]*psi311*p6101)*p5010)*p3000)*p200]
\end{alltt}
